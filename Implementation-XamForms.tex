\documentclass[main.tex]{subfiles}
\begin{document}

Our initial plan was to use the functionality of Xamarin Forms in order to create an application that can be used on both iOS and Android as well as potentially other devices such as windows phone. The appeal of Xamarin Forms is that is provides a hardware independent framework that allows the building of application interfaces using an Extended Application Markup Language (XAML) which has similarities to HTML and other markup languages. This allows for language independent development that is well suited for graphical interface design. Xamarin Forms still requires the use of C# shared code for most advanced features and in order to link components together, however coupled with the application independent nature of the shared code, theoretically, one should be able to create applications completely independently of any single platform that relies on the shared C# code for the back-end and Xamarin forms for the interface.

[insert pic of XamForms here]

\subsection{Issues with Xamarin Forms}

Xamarin Forms is a relatively new development lead by the team at Xamarin and as such, it is still clearly in its infancy in terms of usability and scope of usefulness. While simple applications that only require basic interface functionality such as lists, simple images and tables are easily created, some advanced concepts are still required to be programmed with architecture specific code, and while Xamarin Forms does account and allow for this, it is somewhat limited when compared to using full native code. In our eyes, using Xamarin Forms got to a point where we were simply implementing most of our desired functionality using architecture specific code and where we figured that simply using the native code base for each of our desired platforms would be an easier option in terms of ease of development.

[insert pic of XamForms iOS or android specific lines here]

\subsection{Thoughts on Xamarin Forms}
While Xamarin Forms is an impressive and potentially very useful concept, it is still in its early years and has much growth to do before it can be seriously considered as a viable alternative to native code, (at least for advanced applications). This is though no fault of the developers however as the scope and difficulty of implementing something like Xamarin Forms is truly commendable and will take many, many man hours and much effort in attempting such an advanced framework. It currently shows great potential and there are many examples of how it can be correctly used in order to many a variety of applications and functionality that will work across multiple platforms.

\end{document}