\documentclass[main.tex]{subfiles}
\begin{document}

\section{Xamarin Forms Implementation}

Our initial plan was to use the functionality of Xamarin Forms in order to create an application that can be used on both iOS and Android. The appeal of Xamarin Forms is that is provides a hardware independent framework that allows the building of application interfaces using an Extended Application Markup Language (XAML) which has similarities to HTML and other markup languages. This allows for language independent development that is well suited for graphical interface design. Xamarin Forms still requires the use of C\# shared code for most advanced features and in order to link components together. However, coupled with the application independent nature of the shared code, theoretically, one should be able to create applications completely independently of any single platform that relies on the shared C\# code for the back-end and Xamarin forms for the interface.

\subsection{Issues with Xamarin Forms}

Xamarin Forms is a relatively new development lead by the team at Xamarin and as such, it is still clearly in its infancy in terms of usability and overall scope. While simple applications that only require basic interface functionality such as lists, simple images and tables are easily created, some advanced concepts are still required to be programmed with architecture specific code, and while Xamarin Forms does account and allow for this, it is somewhat limited when compared to using full native code. Certain simple actions, such as dragging an image around the screen and overlaying images on top of one another, were either very difficult or impossible to implement within the current version of Xamarin Forms.

\subsection{Thoughts on Xamarin Forms}
While Xamarin Forms is an impressive and potentially very useful concept, it is still in its early years and has much scope for improvement before it can be seriously considered as a viable alternative to native code, at least for advanced applications. This is through no fault of the developers however as the scope and difficulty of implementing something like Xamarin Forms is truly commendable and will take many, many man hours and much effort in attempting such an advanced framework. It currently shows great potential and there are many examples of how it can be correctly used in order to create a variety of applications and functionality that will work across multiple platforms.

\subsection{Transition to Shared Project}

Given these difficulties, we shifted work over to a Xamarin shared project from the Forms project. This gave us access to native user interface components of both iOS and Android, and made the process of executing our design ideas much more straight forward, without altering any of the shared code (step detection and path finding for example).

\end{document}