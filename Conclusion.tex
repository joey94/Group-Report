\documentclass[main.tex]{subfiles}
\begin{document}
    \chapter{Conclusions and Future Work}
        \section{Conclusions}
            Conclusions will go here
        \section{Future Work}
            Despite the significant progress made over the course of this project, there are obviously some areas in which there is room for
            further research, improvement, and extension to be carried out. The most significant of these is clearly the completion of an Android version of the
            Navigator software that is on par with the current iOS iteration of the application. As discussed earlier in this report when comparing
            the two platforms, unresolvable memory issues due to how Xamarin Studio handles garbage collection for Android applications was the
            primary factor in preventing the completion of this version of the software. In light of this issue, future work could consist of further research
            into solving this specific problem, if such a solution exists. If this is not possible, then it may be prudent to consider developing an Android version
            of the program without the use of Xamarin Studio (a Java based Android application, for example) in order to circumvent the garbage collection problems. This is
            not an ideal solution, however, since Xamarin Studio facilitates cross-platform mobile development, reducing the amount of maintenance required when developing
            for multiple platforms.
            
            Another potential extension to the application is the potential usage of the WiFi system present in most modern smartphones in order to make use
            of WiFi access points to reset the error in the current location of the user. For reasons discussed previously (mostly due to unhelpful hotspot positioning within
            DCS), this functionality could not be included in the application as orginally planned. The addition of such a subsystem to the application has good potential to
            improve the inference of the current location of the user of the software, assuming that their device is WiFi capable. Clearly, such a system would require
            rigorous testing and fine-tuning in order to be accurate, which is why DCS was not a suitable trial location due to its WiFi hotspot positioning. Therefore, 
            any attempts to extend the application to add this functionality should be tested in an appropriate setting, to ensure acceptable progress and results.
            
            Finally, along the same lines as the previous suggestion, other types of landmarks could also be exploited in order to improve user location inference. This
            was not successfully implemented over the course of this work as, after investigating the associated actions (opening of doors, usage of elevators/stairs etc.),
            the collected sensor readings did not allow the implementation of such a system in a way that matched our initial requirements as mentioned in the previously
            created project specification. It is possible, however, that further research may uncover some pattern within the readings that would allow for the implementation
            of this subsystem as originally desired.
\end{document}