\documentclass[main.tex]{subfiles}
\begin{document}

\chapter*{Introduction}

Navigation is the process of accurately determining a person's location and then planning out movements through possible paths to a desired destination. Historically, this has been a problem that has been tackled mainly for the case of exterior navigation. From the more primitive use of stars on sea voyages hundreds of years ago, to the use of technology today, exterior navigation has been solved in large part through the use of satellite navigation systems. The best known and most used of these systems is the Global Positioning System (GPS), navigation technology that is used in almost every sea vessel, plane and motor vehicle that exists today.\\

The development of smartphones has resulted in many of them also being fitted with a GPS unit, allowing navigation to be available to anyone at anytime. Issues exist however with GPS signals penetrating through buildings for use within indoor environments. Indeed, even if signals were able to make their way indoors, the margins of error associated with GPS technology (running into the tens of metres) means that this would likely prove to be ineffective for navigation anyway, considering the smaller overall distances involved when compared to exterior navigation.\\

Despite this, a similarly comprehensive alternative to GPS for interior navigation has yet to be devised.

\section*{Motivation}

The primary motivation for this project lay in the fact that indoor navigation is a problem that still remains relatively unsolved. Despite a wide range of possible solutions making use of different technology, no solution can claim the same kind of universal acceptence that GPS has within the world of exterior navigation. This is especially true for a solution that makes exclusive use of smartphones. As far as we are aware, there is no low cost, user friendly application for smartphones that is able to accurately navigate a user through indoor environments. As such, we were attracted by the prospect of being able to contribute in our own way to the advancement of technology by tackling such an open problem.\\

A solution would also have a myriad of real world applications. Users within large, unfamiliar public buildings could be directed to specific rooms with indoor navigation. Tourists that are visiting hisotrical sites or museums may use such an application to be directed along tour paths tailor made to their specific interests. Users in shopping centres could also be directed to desired stores, with user routing data possibly being used to inform research on consumer behaviour and influence the design of better shopping centres or marketing campaigns. So not only is the problem of effective indoor navigation through the use of smartphones an open one, the pursuit of a solution can also be deemed a worthwhile endevour.

\section*{Objectives}

For this project we proposed the development of a smartphone application that made use a the smartphone in order to track and navigate a user within an indoor environment. For the purposes of the project, the indoor environment was defined as the University of Warwick's Computer Science Department building. The following objectives were set:\\

\begin{itemize}
\item Inertial Sensors - The application should make exclusive use of a smartphone's intertial sensors in order to track a user's movement. This means not considering solutions that involve use of wifi signals or anything other than the movements of the user. This means that the potential solution would be extendable to any building, provided a floor plan exists. 

\item Accuracy - The navigation should be accomplished to a reasonable degree of accuracy. Anything more than a few metres could result in user's being directed to the wrong location. The application should be able to guide users to a correct destination at a consistent level.

\item User Interface - An effective solution would have an intuitive and attractive user interface. The application should display the user's location and the path that must be followed in an appealing and easy to read manner. The floor plans used need to be carefully designed and the interface should feel professionally crafted and not like a proof of concept placeholder.

\end{itemize}

\end{document}
