\documentclass[main.tex]{subfiles}
\begin{document}

\chapter{Management}

The project group consisted of six members, these were Pedram Amirkhalili, Joseph Benrimoj, Dominic Brown, Varun Golani, Robert Hubinsky and Daniel Seabright. At the onset of the project, Joseph Benrimoj was designated as project manager by way of vote amongst the group.

This initial vote set a trend that continued throughout the project, with democratic votes playing a major part in how work was carried out. Considering the fact that all group members shared an equal stake in the ultimate evaluation and grading of the project, it was felt that all group members had to be happy with both the direction and progress of the project. In addition, it was oftentimes only through unanimous approval that decisions were made. This had the added benefit of having conflicting viewpoints result in prolonged discussions about project technicalities and possible design choices. Ultimatively, this was a positive, unintended by product of this meeting format, forcing us to go through chains of thought that we may have not otherwise. As with all traditional project teams, any outstanding disputes would be resolved by the project manager.

Considering the small team size and relative inexperience of most group members with development kit (Xamarin), target platforms (iPhone and Android) and the C\# programming language, it was decided that an iterative development model would be best.

Iterative development involves initial implementations of smaller subsets of software requirements, iteratively enhancing ``the evolving versions until the complete system is developed and ready to be deployed'' [FIX REF - 1]. At each subsequent iteration after the initial software subset implementation, design modifications are made and new functionality added and refined. This is the recommended development model when a new technology is being simultaneously learnt and used by a development team, and carries with it several advantages.\\

Principal amongst these is that a working model of the system is available form early on in development. This is vital when wishing to determine whether any functional or design flaws exist, allowing these to be caught early and rectified at the earliest opportunity. Indeed, as has been explained, this was the case within the project, with the iterative development process helping in identifying the problem earlier than would otherwise have been possible.

Testing also becomes easier. Each iteration of development necessarily requires substantial testing to occur, with most of the those tests being on work carried out from the last iteration. This means tests occur periodically and in smaller grouped quantities, making them more manageable, especially for a small team.

Furthermore, as has been alluded to, the evolutionary nature of the project and software matches the gradual evolution of the team's skills in the new technology being used, complimenting the learning process and permitting work to take place concurrently.

Despite these points, this model makes project management slightly more complex, with greater emphasis on time management and team communication. This fell principally on the project manager, whose job it was to arrange team meetings and facilitate communication between team members.

Communications occured primarily via the use of social media. Announcements about meetings would be placed within a group chat that all members had access to. Here, the project manager would present possible times for meetings with group members all announcing their availability at those times. Meetings would than be arranged once all group members could attend. These face to face meetings occured once a week and were where the majority of the major project decision would take place. Here issues would be dicussed (such as the work conducted over the previous week) and major decision taken with a vote. On the basis of these decision, work would then be assigned amongst group members.

When group members were not all on site in order to carry out the face to face meetings, these would be conducted over skype. Skype calls were also used for work that was done jointly by group members on the same task.
 
 Group roles evolved as work progressed, with these ultimately settling into the following:
 
 iOS development - Joseph and Daniel.\\
 Android Development - Dominic and Robert. \\
 Floorplan Curation - Pedram. \\
 Floorplan graph extraction - Robert. \\
 Step Detection - Varun and Pedram.\\

Work that needed doing would usually be distributed in accordance with these roles, however considering the interconnectedness of components, most group members have had a hand in all aspects of the project.

Communication with the project supervisor occured primarily via email, with face to face meetings occuring on average once every two weeks.

Version control was done using github, with all group members working on the same repository and in most cases working on the same project branch at all times.

\section{Risk Analysis}
    As with all large projects, a risk analysis was conducted. This is shown in table \ref{tab:risk}.
    \begin{table}[h]
        \centering
        \caption{Risk analysis table}
        \label{tab:risk}
        \begin{tabularx}{\textwidth}{|c|c|c| X |}
            \hline 
            \textbf{Risk} & \textbf{Likelihood} & \textbf{Severity} & \textbf{Mitigations} \\ 
            \hline \hline
            Code Loss & Low & High &
            \begin{itemize}[leftmargin=*]
                \item Store code in online GitHub repository
                \item Maintain multiple hard copies on external media
                \item Regularly commit working software changes
            \end{itemize}
            \\ 
            \hline 
            Delays in progress & Moderate & Moderate &
            \begin{itemize}[leftmargin=*]
                \item Timetable Gantt chart carefully
                \item Dedicate an appropriate amount of time to the project per week
                \item Create timetable with contingency in mind in order to reduce the impact of delays
            \end{itemize}        
            \\ 
            \hline 
            Team member illness & Very Low & Moderate &
            \begin{itemize}[leftmargin=*]
                \item Where necessary, work can be redistributed to other members of the team who are familiar with the specific task
                \item Timetabling contingency can also reduce the impact of this risk
            \end{itemize}        
            \\ 
            \hline 
            Inability to obtain licenses & Moderate & High &
            \begin{itemize}[leftmargin=*]
                \item Use open source software where possible
            \end{itemize}        
            \\ 
            \hline 
            Device breakage & Low & Moderate &
            \begin{itemize}[leftmargin=*]
                \item Some testing can be carried out using emulation software
                \item The team has access to multiple Android and iOS devices
            \end{itemize}        
            \\ 
            \hline 
        \end{tabularx}
    \end{table}

\section{Contingency Planning}

As with any project, contingency planning is an important component of good project management. From the onset of our decision to pursue a Xamarin Forms application as a solution for our project, we knew that we required an effective contingency plan to mitigate against this option not working to our satisfaction. Attracted by the prospect of a shared interface across all iPhone and Android devices (as advertised as possible by the Xamarin website), we were nevertheless sceptical about it's ability to handle the more complex user interface requirements of the final application.

With this in mind, we were satisfied with the ability to transfer over to a xamarin shared project form a xamarin forms project. A Xamarin shared project would use the same `shared code', that is code related to step detection, path finding etc. And make use of native components of each platform in order to construct the user interface.

Given the issues that were ultimately encountered with the Xamarin Forms project, this proved to be a shrewd piece of contingency planning, allowing the project to be realligned to a Xamarin shared project within a period of a week or two and for work to continue effectively from that point.

\section{References}

REF - 1 http://www.tutorialspoint.com/sdlc/sdlc\_iterative\_model.htm

\end{document}
