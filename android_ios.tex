\documentclass[main.tex]{subfiles}
\begin{document}
    \section{Analysis of Android vs iOS}
        As the application created over the course of this project was developed for both the Android and iOS platforms, it is prudent to discuss
        the respective benefits and drawbacks of each system with regards to various factors.
        \subsection{Strengths of the Android Platform}
            Whilst there were many problems encountered throughout the project with the Android version of the Navigator application, there were also several
            positives that were associated with developing for the Android platform.
            
            The first of these is that the Android operating system is available for a wide range of devices, including both smartphones and tablet based devices. This allows
            the final application to be tested on a variety of devices, which is particularly useful to us considering that the Android sub-team all possessed differing smartphones
            and tablets. In the event that the application is actually released to the public, this property allows the application to cover for a huge potential user base. This is further
            improved by the backwards compatibility between different iterations of the Android operating system. The level of backwards compatibility is determined by the
            value of the minimum SDK version set in the file \texttt{AndroidManifest.xml}. In this work, the minimum SDK version and target version were both the same, namely
            the Ice Cream Sandwich Android release, API level 15.
            
            The other major benefit of developing for Android is the open source nature of the platform itself. In addition to the operating system, there are many well supported
            open source libraries/extensions available, such as the popular `Mono for Android' (sometimes referred to as monodroid by developers). This is useful as there are many
            situations where specific functionality is not provided by the default Android development environment. Due to the popularity of open  source, however, it is highly likely
            that there is a freely available implementation of the desired functionality available through services such as GitHub, reducing the complexity of the development process.
            This was almost the case with implementing proper
            image scrolling for the floor plan screen, as Mono for Android had some options available. Unfortunately, they were not used due to not fulfilling all of our requirements.
            We did however make use of an open source version of the Recast algorithm.
        \subsection{Drawbacks of the Android Platform}
            Over the course of the project we also came across several problems whilst implementing the Android version of our application. One of the most frustrating and time
            consuming issues encountered was the lack of abstraction provided for dealing with and accessing device sensor readings. For example, obtaining a correct and usable
            azimuth value was an arduous task (as detailed earlier in this document), requiring the use of specialised classes and mathematics. This is in stark contrast to iOS, which
            simply allows these values to be accessed through the location manager system, an approach that is far cleaner and more intuitive than the Android alternative. Another issue
            with sensor access on Android is that the API gives the developer the ability to subscribe to the sensors without the use of a view. This causes problems when a given view is
            paused as you must re-subscribe to the sensors in use, which can be inconvenient at times. It also has the additional effect of requiring the developer to duplicate code in
            situations where data from two different views needs to be processed in the exact same manner.
            
            Whilst mentioned as a positive earlier on in this section, there are also downsides to the ability of Android to be developed and deployed for a large variety of devices. The 
            primary drawback here is that this means the Android operating system cannot be optimised for specific hardware (as the hardware in question cannot be known ahead 
            of time), meaning that there are some potential performance problems. However, iOS does not suffer from this issue due to it only being deployed on Apple devices such as
            the iPhone. Therefore, in terms of overall performance, it is likely that iOS has a slight advantage over the Android platform.
            
            Another large problem with the Android system was briefly touched on earlier in this section when discussing open source: there are many features that would be considered
            fundamental to smartphone usage that are simply not natively implemented for Android. The most problematic of these was the lack of default functionality to properly
            scroll and zoom large bitmap images (This required the development of a custom tool, discussed earlier). In contrast to this, the iOS image view available through
            Xamarin Studio handled this by default.
            
            There was also a major issue encountered that ultimately prevented the completion of the Android version of the application to the same level as that of the iOS version
            in terms of the user interface. This problem was the inability to correctly free certain areas of allocated memory, eventually causing crashes due to memory leaks. This was
            caused by the Xamarin Android platform freeing its own reference to the memory, but failing to ensure that the operating system actually released the memory itself
            (as Android runs its own garbage collection programs). This led to edge cases where, given a high screen refresh rate due to a more complex user interface, the application
            runs out of memory and subsequently crashes. Unfortunately, a solution to this problem could not be found in the limited amount of time remaining
            so our efforts then focused on preparing the iOS version of the application for delivery, seeing as it was not subject to the same limitations.
\end{document}